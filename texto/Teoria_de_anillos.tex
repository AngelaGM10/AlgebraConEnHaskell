\chapter{Teoría de Anillos en Haskell}
En este capítulo daremos una breve introducción a los conceptos de teoría de anillos en Haskell. Para ello haremos uso de los módulos. Un módulo de Haskell es una colección de funciones, tipos y clases de tipos relacionadas entre sí. Un programa Haskell es una colección de módulos donde el módulo principal carga otros módulos y utiliza las funciones definidas en ellos. Así distribuiremos las secciones y partes del código que creeamos necesario en diferentes módulos. Nos centraremos principalmente en las notas sobre cómo definir los conceptos mediante programación funcional y teoría de tipos.

\section{Anillos}
Comenzamos dando las primeras definiciones y propiedades básicas sobre anillos para posteriormente introducir los anillos conmutativos. Creamos el primer módulo \texttt{TAH}\entrada{TAH}

\section{Anillos Conmutativos}
Para espeficar los anillos commutativos crearemos un nuevo módulo en el que importaremos el módulo anterior, el nuevo módulo será \texttt{TAHCommutative}\entrada{TAHCommutative}
\section{Dominio de integridad y Cuerpos}
Dada la noción de anillo conmutativo podemos hablar de estructuras algebraicas como dominio de integridad y cuerpo. Comenzamos por el módulo \texttt{TAHIntegralDomain}\entrada{TAHIntegralDomain} Ahora podemos implementar la especificación de la noción de cuerpo en el módulo \texttt{TAHField}\entrada{TAHField}

\section{Ideales}
En esta sección introduciremos uno de los conceptos importantes en el álgebra conmutativa, el concepto de ideal. Dado que solo consideramos anillos conmutativos, la propiedad multiplicativa de izquierda y derecha son la misma. Veamos su implementación en el módulo \texttt{TAHIdeal}\entrada{TAHIdeal}


