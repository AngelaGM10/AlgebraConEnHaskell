\chapter{Teoría de Anillos en Haskell}
En este capítulo daremos una breve introducción a los conceptos de teoría de anillos en Haskell para ello haremos uso de los módulos. Un módulo de Haskell es una colección de funciones, tipos y clases de tipos relacionadas entre sí. Un programa Haskell es una colección de módulos donde el módulo principal carga otros módulos y utiliza las funciones definidas en ellos para realizar algo. Así distribuiremos las secciones y partes del código que creeamos necesario en diferentes módulos. Nos centraremos principalmente en las notas sobre cómo definir los conceptos en la programación funcional y teoría de tipos.

\section{Anillos}
Comenzamos dando las primeras definiciones y propiedades básicas que tiene un anillo para posteriormente introducir los anillos conmutativos creamos el primer módulo \texttt{TAH}\entrada{TAH}

\section{Anillos Conmutativos}
Para describir los anillos commutativos necesitamos un nuevo módulo \texttt{TAHCommutative}\entrada{TAHCommutative}
\section{Dominio de integridad y Cuerpos}
Dadas las nociones de anillos conmutativos podemos introducir dos conceptos básicos dominio de integridad y cuerpos, comenzamos por el módulo \texttt{TAHIntegralDomain}\entrada{TAHIntegralDomain} Ahora podemos implementar las especificaciones de la noción de cuerpo en el módulo \texttt{TAHCuerpo}\entrada{TAHCuerpo}

\section{Ideales}
El concepto de ideales es muy importante en el álgebra conmutativa. Son
generalizaciones de muchos conceptos de los enteros. Dado que solo consideramos anillos conmutativos, 
la propiedad multiplicativa de izquierda y derecha son la misma. Veamos su implementación en el módulo \texttt{TAHIdeal}\entrada{TAHIdeal}
y seguidamente daremos las nociones de unos anillos que son especialmente relevantes para la construcción matemática. Lo veremos implementado en el módulo \texttt{TAHStronlyDiscrete}\entrada{TAHStronglyDiscrete}

