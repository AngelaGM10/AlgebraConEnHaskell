\chapter{Teoría de anillos en Haskell}\label{sec:anillosHas}

En este capítulo se muestra cómo definir la teoría de anillos en Haskell. Los anillos se pueden definir de forma compacta en grupos y monoides, pero daremos unas series de definiciones que los definen de forma más rigurosa.\\

Comenzamos dando las primeras definiciones y propiedades básicas que tiene un anillo en el módulo \texttt{TAH} 
\entrada{TAH}

Para describir los anillos commutativos necesitamos un nuevo módulo \texttt{TAHCommutative}\entrada{TAHCommutative}

Dentro de este último módulo podemos definir el dominio de integridad, pues se necesita que el anillo sea conmutativo, usaremos el módulo \texttt{TAHIntegralDomain}\entrada{TAHIntegralDomain}

Utilizaremos un nuevo módulo para dar la definición de cuerpo que se encuentra en \texttt{TAHCuerpo}\entrada{TAHCuerpo}
%%% Local Variables:
%%% mode: latex
%%% TeX-master: "TFG"
%%% End:
