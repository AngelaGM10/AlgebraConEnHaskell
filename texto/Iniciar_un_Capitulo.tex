\chapter{Como empezar con Emacs en Ubuntu}\label{sec:InCapHas}

En este capítulo se hace una breve explicación de conceptos básicos para empezar a redactar un documento a LaTeX en Emacs y con Haskell a la vez, así como ir actualizando los archivos junto con la plataforma Github. Comenzaremos explicando como realizar la instalación de Ubuntu 16.04 en un PC con windows 10.\\

\subsection{Instalar Ubuntu 16.04 junto a windows 10}

Para realizar la instalación de Ubuntu junto a windows necesitaremos los siguientes programas:

 
\subsection{Iniciar un Capítulo}

\textbf{Paso 1:}\\
Abrimos el directorio desde Emacs con Ctrl+x+d y accedemos a la carpeta de texto para crear el archivo nuevo .tex sin espacios.

\textbf{Paso 2:}\\
Hacemos lo mismo pero en la carpeta código y guardamos el archivo con la abreviatura que hemos usado en el .tex, el archivo lo guardamos como .lhs para tener ahí el código necesario de Haskell. 

\textbf{Paso 3:}\\
Al acabar el capitulo hay que actualizar el trabajo para que se quede guardado, para ello nos vamos a archivo que contiene todo el trabajo que en nuestro caso se llama 'TFG.tex' importante coger el de la extensión .tex, nos vamos a la zona donde incluimos los capitulos y usamos el comando de LaTeX con el nombre que le dimos en la carpeta de texto:\\

\ include {'nombre sin el .tex'}



\subsection{Abreviaciones de Emacs:}\\

La tecla ctrl se denominara C y la tecla alt M, son las teclas mas utilizadas, pues bien ahora explicamos los atajos más importantes y seguiremos la misma nomenclatura de la guía para las teclas:\\

ctrl es llamada C y alt M\\

Para abrir o crear un archivo:\\
C + x + C + f\\

Para guardar un archivo:\\
C + x + C + s\\

Para guardar un archivo (guardar como):\\
C + x + C + w\\

Si abriste mas de un archivo puedes recorrerlos diferentes buffers con\\
C + x + ← o →\\

Emacs se divide y maneja en buffers y puedes ver varios buffers a la vez (los buffers son como una especie de ventanas).\\

Para tener 2 buffers horizontales:\\
C + x + 2\\

Para tener 2 buffers verticales (si hacen estas combinaciones de teclas seguidas verán que los buffers se suman):\\
C + x + 3\\

Para cambiar el puntero a otro buffer:\\
C + x + o\\

Para tener un solo buffer:\\
C + x + 1\\

Para cerrar un buffer:\\
C + x + k\\

Si por ejemplo nos equivocamos en un atajo podemos cancelarlo con:\\
C + g\\

Para cerrar emacs basta con:\\
C + x + C + C\\

Para suspenderlo:\\
C + z\\

Podemos quitar la suspensión por su id que encontraremos ejecutando el comando:\\

jobs\\

Y después ejecutando el siguiente comando con el id de emacs:\\

fg\\

Escribimos shell y damos enter. \\


\subsection{Push and Pull de Github con Emacs}

Vamos a mostrar como subir y actualizar los archivos en la web de Github desde la Consola (o Terminal), una vez configurado el pc de forma que guarde nuestro usuario y contraseña de Github. Lo primero que debemos hacer es abrir la Consola:\\

Ctrl+Alt+T\\

Escribimos los siguientes comandos en orden para subir los archivos:\\

cd 'carperta en la que se encuentran las subcarpetas de texto y codigo'/ \\

git add .  (de esta forma seleccionamos todo)\\

git commit -m 'nombre del cambio que hemos hecho'\\

git push origin 'master'\\


Para descargar los archivos hacemos lo mismo cambiando el último paso por:\\

git pull origin 'master'\\




El contenido de este capítulo se encuentra en el módulo \texttt{ICH} 
\entrada{ICH}.

%%% Local Variables:
%%% mode: latex
%%% TeX-master: "TFG"
%%% End: