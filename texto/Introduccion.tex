\addcontentsline{toc}{chapter}{Introducción}\label{sec:introHas}
\chapter*{Introducción}

Este trabajo tiene como objetivo implementar conceptos de álgebra constructiva en Haskell, desde la noción de anillo hasta la de anillo coherente. El camino a seguir consiste en desarrollar los siguientes puntos:
\begin{itemize}
\item Familiarización con las funciones predefinidas en Haskell. Aquí aportaremos las funciones básicas de Haskell con el fin de facilitar la lectura del código, pues muchas de ellas son utilizadas a lo largo del trabajo. Algunas de ellas las podemos encontrar en la librería \href{http://hackage.haskell.org/package/base-4.11.1.0/docs/Data-List.html}{\texttt{Data.List}}  de Haskell.

\item Implementación de la teoría de anillos en Haskell. Introduciendo la noción de anillo, para posteriormente incluir los anillos conmutativos, dominio de integridad, cuerpos e ideales. Hemos definido los conceptos mediante programación funcional (\cite{Alonso-15b} y \cite{Hutton-16}) y teoría de tipos (\cite{Alonso-16a} y \cite{aprendehaskell}). De tal forma, que mediante las instancias podemos trabajar con cualquier anillo en concreto, como el anillo de los enteros, \mathbb{Z}. Para la elaboración de este capítulo se ha utilizado el material de la asignatura ``Estructuras Algebraicas'' del Departamento de Álgebra de la facultad de Matemáticas, ``Tema 3: Anillos'' (\cite{Algebra-15}). También se ha utilizado la tesis ``Constructive Algebra in Functional Programming and Type Theory'' de Anders Mörtberg (\cite{tesis}).

\item Librería de vectores y matrices sobre anillos en Haskell. Tomando como referencia la actual librería de matrices de la que Haskell dispone, \href{https://hackage.haskell.org/package/matrix-0.3.6.1/docs/Data-Matrix.html}{\texttt{Data.Matrix}}, junto con la tesis de Anders Mörtberg (\cite{tesis}). En este capítulo, comenzamos con las operaciones básicas entre matrices hasta poder aplicar el Método de Gauss, trabajando con vectores en forma de listas y matrices como listas de vectores sobre un anillo. Los conceptos sobre el método de Gauss-Jordan se encuentran en los apuntes de teoría ``Sistemas de ecuaciones lineales'' del Departamento de Álgebra (\cite{gauss}).

\item Finalmente, incorporamos la teoría sobre anillos coherentes en Haskell. El objetivo de este capítulo es poder resolver sistemas de ecuaciones de la forma $\,M\vec{X}=\vec{b}\,$ en anillos fuertemente discretos y coherentes. La primera sección del capítulo contiene la noción de anillo fuertemente discreto. En la siguiente sección, se define la noción de anillo coherente y se demuestran unas proposiciones que nos permiten implementar la resolución de sistemas de ecuaciones homogéneos. Al final del capítulo, trabajamos con anillos fuertemente discretos y coherentes, los cuáles nos permiten resolver sistemas de ecuaciones no homogéneos. Para el desarrollo de este capítulo, se ha hecho uso de las notas de teoría ``Coherent Ring'' de Thierry Coquand (\cite{coherent-10}) y para el código, se ha utilizado la tesis de Anders Mörtberg (\cite{tesis}).

\end{itemize}

Para poder llevar acabo este trabajo he necesitado utilizar varias herramientas informaticas, daré un breve resumen de cada una de ellas:\\

\begin{itemize}

\item \href{https://www.ubuntu.com/}{\textbf{Ubuntu}}\\
Es un sistema operativo de código abierto para ordenadores. Es una distribución de Linux basada en la arquitectura de Debian. Actualmente corre en ordenadores de escritorio y servidores, en arquitecturas Intel, AMD y ARM. Está orientado al usuario promedio, con un fuerte enfoque en la facilidad de uso y en mejorar la experiencia del usuario. Está compuesto de múltiple software normalmente distribuido bajo una licencia libre o de código abierto. Para desarrollar el trabajo hemos utilizado la versión Ubuntu 16.04 que se encuentra en la web oficial de Ubuntu, los pasos para su instalación están en el apéndice \ref{aped.A}. 

\item \href{https://www.latex-project.org/}{\textbf{LaTeX}}  \\
Es un sistema de composición de textos, orientado a la creación de documentos escritos que presenten una alta calidad tipográfica. Por sus características y posibilidades, se usa en la generación de artículos y libros científicos para incluir expresiones matemáticas, entre otros.  $\mathbf{L\!\!^{{}_{\scriptstyle A}} \!\!\!\!\!\;\; T\!_{\displaystyle E} \! X}$ está formado por un gran conjunto de macros de  $\mathbf{T\!_{\displaystyle E} \! X}$, con la intención de facilitar el uso del lenguaje de composición tipográfica ${\displaystyle \mathbf {T\!_{\displaystyle E}\!X} }$. $\mathbf{L\!\!^{{}_{\scriptstyle A}} \!\!\!\!\!\;\; T\!_{\displaystyle E} \! X}$ es software libre bajo licencia LPPL. 

\item \href{https://www.haskell.org/}{\textbf{Haskell}}\\
Es un lenguaje de programación puramente funcional con semánticas no estrictas y fuertemente tipado. Dentro de Haskell, encontramos \textbf{haskell literario}, con el cuál podemos escribir código en el lenguaje de programación de Haskell al mismo tiempo que podemos redactar texto con las utilidades que ofrece $\mathbf{L\!\!^{{}_{\scriptstyle A}} \!\!\!\!\!\;\; T\!_{\displaystyle E} \! X}$. Todo el trabajo está escrito en  \textbf{haskell literario}. Haskell 8.0.2 es la versión que se ha utilizado para realizar el trabajo.

\item \href{https://www.gnu.org/software/emacs/}{\textbf{Emacs}}\\
Es un editor de texto libre, extensible y personalizable. Emacs es el editor que he usado para programar en Haskell. Este dispone de muchas abreviaciones de teclado para facilitar su uso de una forma más rápida, las más utilizadas se pueden ver en el apéndice \ref{aped.B}.

\item \href{https://github.com/}{\textbf{GitHub}}\\
Es una plataforma de desarrollo colaborativo para alojar proyectos utilizando el sistema de control de versiones Git. Se utiliza principalmente para la creación de código fuente de programas. El código de los proyectos alojados en GitHub se almacena de forma pública, aunque utilizando una cuenta de pago, también permite tener repositorios privados. En mi \href{https://github.com/AngelaGM10}{perfil de GitHub} se encuentra el repositorio con el trabajo. Gracias a esta plataforma puedo compartir los avances con mi tutora. Esto es posible porque al crear un repositorio de GitHub, se pueden incluir colaboradores, que podrán añadir y modificar los elementos del repositorio. Para poder subir los archivos a GitHub podemos hacerlo desde la terminal de Ubuntu, los pasos a seguir están en el apéndice \ref{aped.C}.

\end{itemize}