
% TFG.tex
% TFG
% Autor
% Sevilla, 16 de julio de 2016
% =============================================================================

\documentclass[a4paper,12pt,twoside]{book}

%%%%%%%%%%%%%%%%%%%%%%%%%%%%%%%%%%%%%%%%%%%%%%%%%%%%%%%%%%%%%%%%%%%%%%%%%%%%%%
%%  Paquetes adicionales                                                   %%
%%%%%%%%%%%%%%%%%%%%%%%%%%%%%%%%%%%%%%%%%%%%%%%%%%%%%%%%%%%%%%%%%%%%%%%%%%%%%%

\usepackage[utf8x]{inputenc}       % Acentos de UTF8
\usepackage[spanish]{babel}        % Castellanización.
\usepackage[T1]{fontenc}           % Codificación T1 con European Computer
                                   % Modern.  
\usepackage{graphicx}
\usepackage{fancyvrb}              % Verbatim extendido
\usepackage{makeidx}               % Índice
\usepackage{amsmath}               % AMS LaTeX
\usepackage{amsthm} 
\usepackage{latexsym}
\usepackage[colorinlistoftodos
           , backgroundcolor 
= yellow
           , textwidth = 4cm
, shadow
           , spanish]{todonotes}
% Fuentes
\usepackage{mathpazo}              % Fuentes semejante a palatino
\usepackage[scaled=.90]{helvet}
\usepackage{cmtt}
\renewcommand{\ttdefault}{cmtt}
\usepackage{a4wide}
% \usepackage{xmpincl}               % Incluye metadato de licencia CC.

% Tikz
\usepackage{tkz-berge}
\usetikzlibrary{shapes,trees}
%añadidos por mi:
\usepackage{parskip}
\usepackage{cite} % para contraer referencias

% Desactivar <,> cuando se hacen dibujos con tikz.
\tikzset{
every picture/.append style={
  execute at begin picture={\deactivatequoting},
  execute at end picture={\activatequoting}
  }
}

% Márgenes
\usepackage[margin=1in]{geometry}

% Control de espacios en la tabla de contenidos:
\usepackage[titles]{tocloft}
\setlength{\cftbeforechapskip}{2ex}
\setlength{\cftbeforesecskip}{0.5ex}
\setlength{\cftsecnumwidth}{12mm}
\setlength{\cftsubsecindent}{18mm}

% Control de listas
% Elimina espacio entre item y párrafo y coloca item en el margen izquierdo
% \usepackage{enumitem}
% \setlist[enumerate,itemize]{noitemsep, nolistsep, leftmargin=*}

\usepackage{minitoc}

% Doble espacio entre líneas
\usepackage{setspace}
\onehalfspacing


% \linespread{1.05}                % Distancia entre líneas
\setlength{\parindent}{5mm}        % Indentación de comienzo de párrafo
\deactivatetilden                  % Elima uso de ~ para la eñe
\raggedbottom                      % No ajusta los espacios verticales.

\usepackage[%
  pdftex,
  pdfauthor={Eduardo Paluzo},%
  pdftitle={LPO en Haskell},%
  pdfstartview=FitH,%
  bookmarks=false,%
  colorlinks=true,%
  urlcolor=blue,%
  unicode=true]{hyperref}      

\usepackage{tikz}

%%%%%%%%%%%%%%%%%%%%%%%%%%%%%%%%%%%%%%%%%%%%%%%%%%%%%%%%%%%%%%%%%%%%%%%%%%%%%%
%%  Cabeceras                                                              %%
%%%%%%%%%%%%%%%%%%%%%%%%%%%%%%%%%%%%%%%%%%%%%%%%%%%%%%%%%%%%%%%%%%%%%%%%%%%%%%

\usepackage{fancyhdr}

\addtolength{\headheight}{\baselineskip}

\pagestyle{fancy}

\cfoot{}

\fancyhead{}
\fancyhead[RE]{\bfseries \nouppercase{\leftmark}}
\fancyhead[LO]{\bfseries \nouppercase{\rightmark}}
\fancyhead[LE,RO]{\bfseries \thepage}

%%%%%%%%%%%%%%%%%%%%%%%%%%%%%%%%%%%%%%%%%%%%%%%%%%%%%%%%%%%%%%%%%%%%%%%%%%%%%%
%%  Definiciones                                                           %%
%%%%%%%%%%%%%%%%%%%%%%%%%%%%%%%%%%%%%%%%%%%%%%%%%%%%%%%%%%%%%%%%%%%%%%%%%%%%%%

\input definiciones
\def\mtctitle{Contenido}

%%%%%%%%%%%%%%%%%%%%%%%%%%%%%%%%%%%%%%%%%%%%%%%%%%%%%%%%%%%%%%%%%%%%%%%%%%%%%%
%%  Título                                                                 %%
%%%%%%%%%%%%%%%%%%%%%%%%%%%%%%%%%%%%%%%%%%%%%%%%%%%%%%%%%%%%%%%%%%%%%%%%%%%%%%

\title{\Huge Título del TFG}
\author{Autor}
\date{\vfill \hrule \vspace*{2mm}
  \begin{tabular}{l}
      \href{http://www.cs.us.es/glc}
           {Grupo de Lógica Computacional} \\
      \href{http://www.cs.us.es}
           {Dpto. de Ciencias de la Computación e Inteligencia Artificial} \\
      \href{http://www.us.es}
           {Universidad de Sevilla}  \\
      Sevilla, 16 de junio de 2016 (Versión de \today)
  \end{tabular}\hfill\mbox{}}




%%%%%%%%%%%%%%%%%%%%%%%%%%%%%%%%%%%%%%%%%%%%%%%%%%%%%%%%%%%%%%%%%%%%%%%%%%%%%%%
%%  Construcción del índice                                                 %%
%%%%%%%%%%%%%%%%%%%%%%%%%%%%%%%%%%%%%%%%%%%%%%%%%%%%%%%%%%%%%%%%%%%%%%%%%%%%%%%

\makeindex

%%%%%%%%%%%%%%%%%%%%%%%%%%%%%%%%%%%%%%%%%%%%%%%%%%%%%%%%%%%%%%%%%%%%%%%%%%%%%%
%%  Documento                                                              %%
%%%%%%%%%%%%%%%%%%%%%%%%%%%%%%%%%%%%%%%%%%%%%%%%%%%%%%%%%%%%%%%%%%%%%%%%%%%%%%

% \includeonly{Introduccion}

% \includexmp{licencia}

\begin{document}

\dominitoc

\begin{titlepage}
 \vspace*{2cm}
  \begin{center}
    {\huge \textbf{Álgebra constructiva en Haskell}}
  \end{center}
  \vspace{4cm}
  \begin{center}
    \leavevmode\includegraphics[totalheight=6cm]{sello.png}\\[3cm]
    {\normalsize Facultad de Matemáticas} \\
    {\normalsize Departamento de Ciencias de la Computación e Inteligencia Artificial}\\
    {\normalsize Trabajo Fin de Grado} \\
  \end{center}
  \begin{center}
    {\large \textbf{Ángela González Martín}}
  \end{center}
  \newpage
 
 \begin{flushright}
   \vspace*{5cm}
   \begin{minipage}{8.45cm}
      \textbf{Agradecimientos}\\

      A mis padres quiero agradecerles la ayuda y apoyo que me han dado
      durante esta bonita etapa. Siempre han estado para mí cuando
      más lo he necesitado, gracias por haberme facilitado el poder 
      estudiar Matemáticas, por ayudarme en los momentos difíciles 
      y por apoyarme en mis decisiones.\\

      A mi tutora María José, por toda la paciencia que ha tenido conmigo y la gran 
      ayuda que me ha brindado para poder realizar este trabajo.\\

      Por último y no menos importante, a todos mis amigos y todas las 
      personas que son importantes para mí que han recorrido este camino 
      conmigo. Ellos han hecho que haya sido una de las mejores experiencias.
       
    \end{minipage}

      \vspace*{7.5mm}
      
      
  \end{flushright}
  \vspace*{\fill}

  \newpage


  
  % \begin{flushright}
  \begin{center}
   \vspace*{5cm}
    \begin{minipage}{14cm}
      El presente Trabajo Fin de Grado se ha realizado en el Departamento de
      Ciencias de la Computación e Inteligencia Artificial de la Universidad de
      Sevilla.

      \vspace*{7.5mm}

      Supervisado por
      % \vspace*{5mm}
    \end{minipage}\par
    María José Hidalgo Doblado
    % \end{flushright}
    \end{center}
  \vspace*{\fill}

  \newpage

  \vspace*{3cm}
  {\huge \textit{Abstract}}

  \vspace{2cm}
  The objective of this proyect is to represent the algebraic structures with a functional programming language. For this, we have maked use of Haskell. Through the use of type classes it has been specified the notion of ring, commutative ring, integral domain, field, ideal, strongly discrete ring, coherent ring and finally strongly discrete and coherent ring. In the last one it is possible to solve systems of equations, for which it has developed a librery of vectors and matrices which elements are in a generic structure (ring, integral domain or field). Furthermore, through an installation process it can get specific examples of the structures represented.
  
\end{titlepage}
\newpage

\input{licencia/licenciaCC}
\newpage 

\tableofcontents
\newpage

\addcontentsline{toc}{chapter}{Introducción}\label{sec:introHas}
\chapter*{Introducción}

Este trabajo tiene como objetivo implementar conceptos de álgebra constructiva en Haskell, desde la noción de anillo hasta la de anillo coherente. El camino a seguir consiste en desarrollar los siguientes puntos:
\begin{itemize}
\item Familiarización con las funciones predefinidas en Haskell. Aquí aportaremos las funciones básicas de Haskell con el fin de facilitar la lectura del código, pues muchas de ellas son utilizadas a lo largo del trabajo. Algunas de ellas las podemos encontrar en la librería \href{http://hackage.haskell.org/package/base-4.11.1.0/docs/Data-List.html}{\texttt{Data.List}}  de Haskell.

\item Implementación de la teoría de anillos en Haskell. Introduciendo la noción de anillo, para posteriormente incluir los anillos conmutativos, dominio de integridad, cuerpos e ideales. Hemos definido los conceptos mediante programación funcional (\cite{Alonso-15b} y \cite{Hutton-16}) y teoría de tipos (\cite{Alonso-16a} y \cite{aprendehaskell}). De tal forma, que mediante las instancias podemos trabajar con cualquier anillo en concreto, como el anillo de los enteros, \mathbb{Z}. Para la elaboración de este capítulo se ha utilizado el material de la asignatura ``Estructuras Algebraicas'' del Departamento de Álgebra de la facultad de Matemáticas, ``Tema 3: Anillos'' (\cite{Algebra-15}). También se ha utilizado la tesis ``Constructive Algebra in Functional Programming and Type Theory'' de Anders Mörtberg (\cite{tesis}).

\item Librería de vectores y matrices sobre anillos en Haskell. Tomando como referencia la actual librería de matrices de la que Haskell dispone, \href{https://hackage.haskell.org/package/matrix-0.3.6.1/docs/Data-Matrix.html}{\texttt{Data.Matrix}}, junto con la tesis de Anders Mörtberg (\cite{tesis}). En este capítulo, comenzamos con las operaciones básicas entre matrices hasta poder aplicar el Método de Gauss, trabajando con vectores en forma de listas y matrices como listas de vectores sobre un anillo. Los conceptos sobre el método de Gauss-Jordan se encuentran en los apuntes de teoría ``Sistemas de ecuaciones lineales'' del Departamento de Álgebra (\cite{gauss}).

\item Finalmente, incorporamos la teoría sobre anillos coherentes en Haskell. El objetivo de este capítulo es poder resolver sistemas de ecuaciones de la forma $\,M\vec{X}=\vec{b}\,$ en anillos fuertemente discretos y coherentes. La primera sección del capítulo contiene la noción de anillo fuertemente discreto. En la siguiente sección, se define la noción de anillo coherente y se demuestran unas proposiciones que nos permiten implementar la resolución de sistemas de ecuaciones homogéneos. Al final del capítulo, trabajamos con anillos fuertemente discretos y coherentes, los cuáles nos permiten resolver sistemas de ecuaciones no homogéneos. Para el desarrollo de este capítulo, se ha hecho uso de las notas de teoría ``Coherent Ring'' de Thierry Coquand (\cite{coherent-10}) y para el código, se ha utilizado la tesis de Anders Mörtberg (\cite{tesis}).

\end{itemize}

Para poder llevar acabo este trabajo he necesitado utilizar varias herramientas informaticas, daré un breve resumen de cada una de ellas:\\

\begin{itemize}

\item \href{https://www.ubuntu.com/}{\textbf{Ubuntu}}\\
Es un sistema operativo de código abierto para ordenadores. Es una distribución de Linux basada en la arquitectura de Debian. Actualmente corre en ordenadores de escritorio y servidores, en arquitecturas Intel, AMD y ARM. Está orientado al usuario promedio, con un fuerte enfoque en la facilidad de uso y en mejorar la experiencia del usuario. Está compuesto de múltiple software normalmente distribuido bajo una licencia libre o de código abierto. Para desarrollar el trabajo hemos utilizado la versión Ubuntu 16.04 que se encuentra en la web oficial de Ubuntu, los pasos para su instalación están en el apéndice \ref{aped.A}. 

\item \href{https://www.latex-project.org/}{\textbf{LaTeX}}  \\
Es un sistema de composición de textos, orientado a la creación de documentos escritos que presenten una alta calidad tipográfica. Por sus características y posibilidades, se usa en la generación de artículos y libros científicos para incluir expresiones matemáticas, entre otros.  $\mathbf{L\!\!^{{}_{\scriptstyle A}} \!\!\!\!\!\;\; T\!_{\displaystyle E} \! X}$ está formado por un gran conjunto de macros de  $\mathbf{T\!_{\displaystyle E} \! X}$, con la intención de facilitar el uso del lenguaje de composición tipográfica ${\displaystyle \mathbf {T\!_{\displaystyle E}\!X} }$. $\mathbf{L\!\!^{{}_{\scriptstyle A}} \!\!\!\!\!\;\; T\!_{\displaystyle E} \! X}$ es software libre bajo licencia LPPL. 

\item \href{https://www.haskell.org/}{\textbf{Haskell}}\\
Es un lenguaje de programación puramente funcional con semánticas no estrictas y fuertemente tipado. Dentro de Haskell, encontramos \textbf{haskell literario}, con el cuál podemos escribir código en el lenguaje de programación de Haskell al mismo tiempo que podemos redactar texto con las utilidades que ofrece $\mathbf{L\!\!^{{}_{\scriptstyle A}} \!\!\!\!\!\;\; T\!_{\displaystyle E} \! X}$. Todo el trabajo está escrito en  \textbf{haskell literario}. Haskell 8.0.2 es la versión que se ha utilizado para realizar el trabajo.

\item \href{https://www.gnu.org/software/emacs/}{\textbf{Emacs}}\\
Es un editor de texto libre, extensible y personalizable. Emacs es el editor que he usado para programar en Haskell. Este dispone de muchas abreviaciones de teclado para facilitar su uso de una forma más rápida, las más utilizadas se pueden ver en el apéndice \ref{aped.B}.

\item \href{https://github.com/}{\textbf{GitHub}}\\
Es una plataforma de desarrollo colaborativo para alojar proyectos utilizando el sistema de control de versiones Git. Se utiliza principalmente para la creación de código fuente de programas. El código de los proyectos alojados en GitHub se almacena de forma pública, aunque utilizando una cuenta de pago, también permite tener repositorios privados. En mi \href{https://github.com/AngelaGM10}{perfil de GitHub} se encuentra el repositorio con el trabajo. Gracias a esta plataforma puedo compartir los avances con mi tutora. Esto es posible porque al crear un repositorio de GitHub, se pueden incluir colaboradores, que podrán añadir y modificar los elementos del repositorio. Para poder subir los archivos a GitHub podemos hacerlo desde la terminal de Ubuntu, los pasos a seguir están en el apéndice \ref{aped.C}.

\end{itemize}

\chapter{Programación funcional con Haskell}\label{sec:progfunHas}

En este capítulo se hace una breve introducción a la programación funcional en
Haskell suficiente para entender su aplicación en los siguientes
capítulos. Para una introducción más amplia se pueden consultar los apuntes de
la asignatura de Informática de 1º del Grado en Matemáticas
(\cite{Alonso-15b}), la web de "¡Aprende haskell por el bien de todos!" (\cite{aprendehaskell}) y la web oficial de Haskell (\cite{haskell}) . Las funciones predefinidas en Haskell que vamos a utilizar en los siguientes capítulos pertenecen, entre otras, a la librería \href{http://hackage.haskell.org/package/base-4.11.1.0/docs/Data-List.html}{\texttt{Data.List}} de Haskell.\\

Con la programación puramente funcional no ejecutamos órdenes, sino más bien, definimos como son las cosas. Al definir una función se le asocia un tipo que determina el comportamiento de la función, esto permite al compilador razonar acerca de el comportamiento, y nos permite ver si una función es correcta o no antes de ejecutarla. De esta forma podemos construir funciones más complejas uniendo funciones simples.\\

Haskell es un lenguaje tipado. Esto significa que todos los ojetos tienen asociado un tipo. También las funciones, por lo que se pueden detectar errores en tiempo de compilación.\\

Vamos a explicar brevemente los conceptos importantes para entender mejor la programación funcional en Haskell:

\begin{itemize}

\item \textbf{Funciones de orden superior}

Las funciones de Haskell pueden tomar funciones como parámetros y devolver funciones como resultado. Una función que hace ambas cosas o alguna de ellas se llama función de orden superior. Es decir, si una función toma como argumento a otra función o devuelve una función como resultado se dice que es una función de orden superior. Por ejemplo, las funciones \texttt{foldr}, \texttt{filter} y \texttt{zipWith} son funciones de orden superior, todas ellas toman a otra función como argumento.\\

\item \textbf{Tipos}

Cada función en Haskell tiene asociada un tipo. Al programar la función no es necesario especificar el tipo, pues Haskell lo puede inferir. Es conveniente especificar el tipo sobre el que queremos utilizar la función, para que al crear nuevas funciones no se produzcan problemas de tipos.\\

Un tipo en Haskell es como un conjunto de elementos que tienen algo en común. Por ejemplo el grupo de los números enteros \mathbb{Z} se representa por el tipo \texttt{Integer}. Los tipos básicos de Haskell son: \texttt{Integer}, \texttt{Int}, \texttt{Char}, \texttt{Bool}, \texttt{Double} y \texttt{Float}.\\

\item  \textbf{Clases de tipos}

Las clases de tipos son una especie de interfaz que define algún tipo de comportamiento. Si un tipo es miembro de una clase de tipos, significa que ese tipo soporta e implementa el comportamiento que define la clase de tipos. Cada clase tiene unas propiedades que se aplican a los elementos que se utilizan estando en dicha clase. Por ejemplo: \texttt{Eq}, \texttt{Show} y \texttt{Ord} son algunas de las clases básicas.\\

Podemos crear nuevas clases de tipos escribiendo \texttt{class Ring a where}, la nueva clase de tipo que hemos definido se llama \texttt{Ring}. La \texttt{a} es la variable de tipo, esta representará el tipo sobre el que se trabajará en la clase \texttt{Ring}. Debajo del \texttt{where} definimos varias funciones. Solo hay que especificar las declaraciones de tipo de las funciones.\\

\item \textbf{Restricciones de clases de tipos}

Para concretar sobre qué propiedades queremos trabajar se puede restringir una clase dentro de otra. Por ejemplo, la función \texttt{group} tiene el siguiente tipo:
\begin{center}
\texttt{group :: Eq a =>  [a] ->  [[a]]}
\end{center}
La función \texttt{group} recibe una lista de elementos de \texttt{a}. Se exige que \texttt{a} sea un tipo con igualdad, es decir, que pertenezca a la clase \texttt{Eq}. Sin embargo la función \texttt{elem} con el siguiente tipo:
\begin{center}
\texttt{elem :: (Foldable t, Eq a) =>  a ->  t a -> Bool}
\end{center}
Esta restringida a dos clases, de esta forma la función \texttt{ elem } tiene las propiedades de ambas clases.\\ 

\item \textbf{Instancias}

Un tipo puede ser una instancia de una clase si soporta el comportamiento de la clase. Por ejemplo:  El tipo \texttt{Int} es una instancia de la clase \texttt{Eq}, ya que la clase de tipos \texttt{Eq} define el comportamiento de elementos que se pueden equiparar, y los números enteros pueden equipararse.\\

También podemos utilizar instancias sobre las clases que hayamos creado. Al declarar una instancia, estamos declarando el tipo sobre el que la clase va a actuar bajo los atributos que se definan. Las funciones que se definan y estén restringidas a la nueva clase, verificará los atributos de esta clase.

\end{itemize}


%%% Local Variables:
%%% mode: latex
%%% TeX-master: "TFG"
%%% End:


%\include{Como_Empezar}

\chapter{Teoría de anillos en Haskell}\label{sec:anillosHas}

En este capítulo se muestra cómo definir la teoría de anillos en Haskell. Los anillos se pueden definir de forma compacta en grupos y monoides, pero daremos unas series de definiciones que los definen de forma más rigurosa.\\

Comenzamos dando las primeras definiciones y propiedades básicas que tiene un anillo en el módulo \texttt{TAH} 
\entrada{TAH}

Para describir los anillos commutativos necesitamos un nuevo módulo \texttt{TAHCommutative}\entrada{TAHCommutative}

Dentro de este último módulo podemos definir el dominio de integridad, pues se necesita que el anillo sea conmutativo, usaremos el módulo \texttt{TAHIntegralDomain}\entrada{TAHIntegralDomain}

Utilizaremos un nuevo módulo para dar la definición de cuerpo que se encuentra en \texttt{TAHCuerpo}\entrada{TAHCuerpo}

\section{Ideales}
El concepto de ideales es muy importante en el álgebra conmutativa. Son
generalizaciones de muchos conceptos de los enteros. Dado que solo consideramos anillos conmutativos, la propiedad multiplicativa de izquierda y derecha son la misma. Veamos su implementación en 

\section{Anillos fuertemente discretos}
Esta sección considerará algunos anillos que son especialmente relevantes para la construcción matemática. Empezamos con el módulo \texttt{TAHStronlyDiscrete}\entrada{TAHStronglyDiscrete}

%%% Local Variables:
%%% mode: latex
%%% TeX-master: "TFG"
%%% End:


\chapter{Vectores y Matrices}\label{sec:matrixHas}
En este capítulo implementaremos los vectores y matrices como listas para poder usarlas en los módulos siguientes. En Haskell existen ya estas librerías pero solo podemos usarlas si nos restringimos al tipo numérico. Por ello, crearemos este módulo con el fin de generalizar los conceptos para anillos sobre cualquier tipo $a$.

\section{Vectores}
Antes de introducir las matrices tenemos que especificar ciertas operaciones y funciones sobre los vectores, pues las filas de las matrices serán vectores, así tendremos la matriz como lista de listas. Damos los conceptos necesarios en el módulo \texttt{TAHVector}\entrada{TAHVector} 

\section{Matrices}
Una vez dadas las funciones y operaciones de los vectores podemos comenzar a implementar las matrices, notesé que cada fila o columna de una matriz puede verse como un vector. Nuestra idea es dar las matrices como una lista de vectores de forma que cada vector será una fila de la matriz. El objetivo de esta sección es implementar el método de Gauss-Jordan con el fin de poder resolver sistemas de la forma $Ax=b$. Lo vemos en el módulo \texttt{TAHMatrix}\entrada{TAHMatrix} 

\chapter{Anillos Coherentes}\label{sec:coherentHas}
Todos los anillos en este capítulo son dominios integrales. Uno de los principales objetivos de las siguientes secciones serán demostrar que los diferentes anillos son coherentes. Esto significa que es posible resolver sistemas de ecuaciones en ello. Antes de comenzar introduciremos la noción de anillo fuertemente discreto.
\section{Anillos Fuertemente Discretos}
En esta breve sección mostraremos la noción de anillo discreto y fuertemente discreto, lo vemos en el módulo \texttt{TAHStronglyDiscrete}\entrada{TAHStronglyDiscrete} 

\section{Vectores y Matrices}
En esta sección implementaremos los vectores y matrices como listas para poder construir los módulos siguientes. En Haskell existen ya estas librerías pero solo podemos usarlas si nos restringimos al tipo numérico. Por ello, crearemos este módulo con el fin de generalizar los conceptos para cualquier tipo de anillo. LLamamos a este módulo como \texttt{TAHMatrix}\entrada{TAHMatrix}

\section{Anillos Coherentes}
En esta sección, nos ayudaremos del módulo de vectores y matrices creado en el capítulo anterior para construir, en Haskell, la noción de anillo coherente con el objetivo de poder resolver sistemas de ecuaciones. Lo vemos en el módulo \texttt{TAHCoherent}\entrada{TAHCoherent} 
%%% Local Variables:
%%% mode: latex
%%% TeX-master: "TFG"
%%% End:

%%%%%%%%%%%%%%%%%%%%%%%%%%%%%%%%%%%%%%%%%%%%%%%%%%%%%%%%%%%%%%%%%%%%%%%%%%%%%%% 
%%  Apéndice                                                         %%
%%%%%%%%%%%%%%%%%%%%%%%%%%%%%%%%%%%%%%%%%%%%%%%%%%%%%%%%%%%%%%%%%%%%%%%%%%%%%%%

\appendix

\include{Como_Empezar}


%%%%%%%%%%%%%%%%%%%%%%%%%%%%%%%%%%%%%%%%%%%%%%%%%%%%%%%%%%%%%%%%%%%%%%%%%%%%%%% 
%%  Bibliografía                                                            %%
%%%%%%%%%%%%%%%%%%%%%%%%%%%%%%%%%%%%%%%%%%%%%%%%%%%%%%%%%%%%%%%%%%%%%%%%%%%%%%%

\addcontentsline{toc}{chapter}{Bibliografía}
\bibliographystyle{abbrv}
\bibliography{TFG}

%%%%%%%%%%%%%%%%%%%%%%%%%%%%%%%%%%%%%%%%%%%%%%%%%%%%%%%%%%%%%%%%%%%%%%%%%%%%%%%
%%  Índice                                                                  %%
%%%%%%%%%%%%%%%%%%%%%%%%%%%%%%%%%%%%%%%%%%%%%%%%%%%%%%%%%%%%%%%%%%%%%%%%%%%%%%%

\addcontentsline{toc}{chapter}{Indice de definiciones}

\printindex

%%%%%%%%%%%%%%%%%%%%%%%%%%%%%%%%%%%%%%%%%%%%%%%%%%%%%%%%%%%%%%%%%%%%%%%%%%%%%%%
%% § Pendientes                                                              %%
%%%%%%%%%%%%%%%%%%%%%%%%%%%%%%%%%%%%%%%%%%%%%%%%%%%%%%%%%%%%%%%%%%%%%%%%%%%%%%%

%\todototoc
%\listoftodos

\end{document}



%%% Local Variables:
%%% mode: latex
%%% TeX-master: t
%%% End:
