\chapter{Anillos Coherentes}\label{sec:coherentHas}
Todos los anillos en este capítulo son dominios de integridad. Uno de los principales objetivos de las siguientes secciones será ver cuando un anillo es coherente. Es decir, que es posible resolver sistemas de ecuaciones en ellos. Antes de comenzar, introduciremos la noción de anillo fuertemente discreto.
\section{Anillos Fuertemente Discretos}
En esta breve sección mostraremos la noción de anillo discreto y fuertemente discreto, lo vemos en el módulo \texttt{TAHStronglyDiscrete}\entrada{TAHStronglyDiscrete} 

\section{Anillos Coherentes}
En esta sección, nos ayudaremos del módulo de vectores y matrices creado en el capítulo anterior para construir, en Haskell, la noción de anillo coherente con el objetivo de poder resolver sistemas de ecuaciones. Lo vemos en el módulo \texttt{TAHCoherent}\entrada{TAHCoherent} 
%%% Local Variables:
%%% mode: latex
%%% TeX-master: "TFG"
%%% End: