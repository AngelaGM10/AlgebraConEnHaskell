\chapter{Anillos Coherentes}\label{sec:coherentHas}
Todos los anillos en este capítulo son dominios integrales. Uno de los principales objetivos de las siguientes secciones serán demostrar que los diferentes anillos son coherentes. Esto significa que es posible resolver sistemas de ecuaciones en ello.
\section{Matrices y Vectores en Haskell}
Antes veremos una breve implementación de vectores y matrices en Haskell con el objetivo de conseguir resolver sistemas de ecuaciones con matrices del tipo $Ax=b$ mediante Gauss-Jordan, lo vemos en el módulo \texttt{TAHMatriz}\entrada{TAHMatriz} 

\section{Anillos Coherentes y Fuertemente Discretos}
Ya podemos resolver sistemas de ecuaciones con matrices sobre anillos conmutativos, ahora daremos las nociones de anillos coherentes, lo vemos implementado en el módulo \texttt{TAHCoherent}\entrada{TAHCoherent}

%%% Local Variables:
%%% mode: latex
%%% TeX-master: "TFG"
%%% End: