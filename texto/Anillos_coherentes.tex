\chapter{Anillos Coherentes}\label{sec:coherentHas}
El objetivo de este capítulo será ver cuando un anillo es coherente y fuertemente discreto. Gracias a esto podremos resolver sistemas de ecuaciones sobre estos anillos. Antes de comenzar, introduciremos la noción de anillo fuertemente discreto.
\section{Anillos Fuertemente Discretos}
En esta breve sección mostraremos la noción de anillo discreto y fuertemente discreto, lo vemos en el módulo \texttt{TAHStronglyDiscrete}\entrada{TAHStronglyDiscrete} 

\section{Anillos Coherentes}
En esta sección, nos ayudaremos del módulo de vectores y matrices creado en el capítulo anterior para construir, en Haskell, la noción de anillo coherente con el objetivo de poder resolver sistemas de ecuaciones. Lo vemos en el módulo \texttt{TAHCoherent}\entrada{TAHCoherent} 

\section{Anillos Coherentes y Fuertemente Discretos }
Dentro de los anillos anillos coherentes, podemos considerar los anillos coherentes fuertemente discretos. Restringiendo la clase de anillos a la clase de fuertemente discreto, nos permite una mayor facilidad a la hora de resolver sistemas. Puesto que, si un anillo es fuertemente discreto y coherente entonces podemos resolver ecuaciones del tipo $AX=b$. Lo vemos en el módulo \texttt{TAHCoherentSD}\entrada{TAHCoherentSD} 


%%% Local Variables:
%%% mode: latex
%%% TeX-master: "TFG"
%%% End: